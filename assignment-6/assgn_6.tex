% Inbuilt themes in beamer
\documentclass{beamer}

% Theme choice:
\usetheme{CambridgeUS}

% Title page details: 
\title{Assignment $6$\\ Probability and random Variables} 
\institute{Indian Institute of Technology Hyderabad}
\author{Shreyas Wankhede}
\date{\today}
\logo{\large \LaTeX{}}


\begin{document}

% Title page frame
\begin{frame}
      \titlepage 
\end{frame} 

% Remove logo from the next slides
\logo{}


% Outline frame
\begin{frame}{Outline}
    \tableofcontents
\end{frame}


% Lists frame

\section{Example Question}
\begin{frame}{Example Question}
\textbf{Papoulis book example 4.5}\\\vspace{5mm}
A telephone call occurs at random in the interval (0, 1). In this experiment. the
outcomes are time distances t between 0 and 1 and the probability that t is between t1
and t2 is given by 
\begin{align}
   P\{t_1\le t \le t_2\}= t_2 - t_1 \nonumber
\end{align}
We define the random variable x such that
\begin{align}
    x(t) = t \hspace{5mm} 0\le t\le 1 \nonumber
\end{align}

\end{frame}

\section{Explanation}
\begin{frame}{Explanation}
 Thus the variable t has a double meaning: It is the outcome of the experiment and the
corresponding value x(t) of the random variable:x. We shall show that the distribution
function F(x) of x is a ramp as in Fig.1. \\
  If $x>1$, then $X(t) \le x$ for every outcome of. Hence,
  \begin{align}
      F(x) =P\{X\le x\} = P\{ 0\le t \le 1\}= P(S) =1 \nonumber
  \end{align}
\end{frame} 

\begin{frame}{figure}
       \begin{figure}
              \includegraphics[width=3in,height=2.4in]{Figure_1.png}
              \caption{fig-1}
              \label{fiure-1}
            \end{figure}
\end{frame}

\begin{frame}
\hspace*{10mm}If $x>1$, then $X(t)\le x$ for every outcome. Hence,\\
\hspace*{20mm}$F(x) = P\{X\le x\} = P\{0\le t \le 1\}= P(S)=1$\\
\vspace*{10mm}   
\hspace*{10mm}If $0 \le x \le 1$, then $X(t)\le x$ for every $t$ in the interval $(0,x)$. Hence,\\
\hspace*{20mm}$F(x) = P\{X\le x\} = P\{0\le t \le x\}=x \hspace*{3mm} 0\le x \le 1 $\\     
\vspace*{10mm}    
\hspace*{10mm}If $x<0$, then $\{X \le x\}$ is the impossible event for $x(t) \ge 0$. Hence,\\
\hspace*{20mm}$F(x) = P\{X\le x\} = 0$ \hspace*{3mm} $x<0$\\              
\end{frame}

\end{document}